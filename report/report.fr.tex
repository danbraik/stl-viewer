\documentclass[10pt]{article}

\usepackage[french]{babel}
\usepackage[utf8]{inputenc}
\usepackage{fancyhdr}
\usepackage{lastpage}
\usepackage{hyperref}

%%%%%%
% Pour mise-en-forme des fichiers Ada
%
% voir exemple en fin de ce fichier.
%
% ATTENTION, requiert encoding utf-8 (voir 2ième "\lstset" ci-dessous)
 
\usepackage{listings}
\lstset{
  morekeywords={abort,abs,accept,access,all,and,array,at,begin,body,
      case,constant,declare,delay,delta,digits,do,else,elsif,end,entry,
      exception,exit,for,function,generic,goto,if,in,is,limited,loop,
      mod,new,not,null,of,or,others,out,package,pragma,private,
      procedure,raise,range,record,rem,renames,return,reverse,select,
      separate,subtype,task,terminate,then,type,use,when,while,with,
      xor,abstract,aliased,protected,requeue,tagged,until},
  sensitive=f,
  morecomment=[l]--,
  morestring=[d]",
  showstringspaces=false,
  basicstyle=\small\ttfamily,
  keywordstyle=\bf\small,
  commentstyle=\itshape,
  stringstyle=\sf,
  extendedchars=true,
  columns=[c]fixed
}

% CI-DESSOUS: conversion des caractères accentués UTF-8 
% en caractères TeX dans les listings...
\lstset{
  literate=%
  {À}{{\`A}}1 {Â}{{\^A}}1 {Ç}{{\c{C}}}1%
  {à}{{\`a}}1 {â}{{\^a}}1 {ç}{{\c{c}}}1%
  {É}{{\'E}}1 {È}{{\`E}}1 {Ê}{{\^E}}1 {Ë}{{\"E}}1% 
  {é}{{\'e}}1 {è}{{\`e}}1 {ê}{{\^e}}1 {ë}{{\"e}}1%
  {Ï}{{\"I}}1 {Î}{{\^I}}1 {Ô}{{\^O}}1%
  {ï}{{\"i}}1 {î}{{\^i}}1 {ô}{{\^o}}1%
  {Ù}{{\`U}}1 {Û}{{\^U}}1 {Ü}{{\"U}}1%
  {ù}{{\`u}}1 {û}{{\^u}}1 {ü}{{\"u}}1%
}

%%%%%%%%%%
% TAILLE DES PAGES (A4 serré)

\setlength{\parindent}{0pt}
\setlength{\parskip}{1ex}
\setlength{\textwidth}{17cm}
\setlength{\textheight}{24cm}
\setlength{\oddsidemargin}{-.7cm}
\setlength{\evensidemargin}{-.7cm}
\setlength{\topmargin}{-.5in}

%%%%%%%%%%
% EN-TÊTES ET PIED DE PAGES

\pagestyle{fancyplain}
\renewcommand{\headrulewidth}{0pt}
\addtolength{\headheight}{0pt}
\addtolength{\headheight}{0pt}
\lfoot{}
\cfoot{}
\rfoot{\footnotesize\sf \thepage/\pageref{LastPage}}
\lhead{\footnotesize\sf }
\rhead{\footnotesize\sf Teide 105} % numéro d'équipe Teide 

%%%%%%%%%%
% TITRE DU DOCUMENT


%%%%%%%

\begin{document}

\begin{center} Compte rendu du projet d'algorithmique 1 : Visualiseur 3D \end{center}
\begin{center} (équipe Teide 105) Baptiste JOSI, Daniel BRAIKEH \end{center}


Le projet consistait en la réalisation d'un logiciel capable d'afficher en perspective les données 3D issues d'un fichier STL.
La version proposée est capable d'afficher le modèle en fils de fer, en triangles pleins, avec plusieurs gestions de la luminosité, et l'affichage des normales. Les fichiers binaires et ASCII sont également supportés.


\section{Implémentation}

\subsection{Implémentation du paquetage {\tt STL}}
Le paquetage {\tt STL} propose une unique fonction de chargement, qui détermine seule le type réel du fichier passé en argument. Ainsi, on est capable de charger un fichier STL Ascii et STL binaire. Afin d'optimiser le chargement ASCII, le parcours du fichier n'est effectué qu'une seule fois. Pour cela, les facettes sont enregistrées dans une structure de donnée de type liste chaînée. Afin de garder une homogénéité avec le reste du programme, la liste est convertie en tableau. Pour le chargement binaire, la solution est plus simple, en effet, le début du fichier indique le nombre de faces à charger.
Le chargement Ascii réalise le modèle d'une machine séquentielle permettant d'obtenir mot après mot. Grâce à des états, nous pouvons charger les différents sommets des faces et même gérer les faces incomplètes. Enfin, les normales des faces sont recalculées à partir des sommets.

\subsection{Implémentation du paquetage {\tt Algebre}}
Le paquetage {\tt Algebre} regroupe les fonctions de rotation de matrices, de produit entre matrices et vecteurs, de produit vectoriel, d'addition et soustraction de vecteurs, de normalisation de vecteur. Nous avons pour cela utilisé la surcharge d'opérateurs.
Le paquetage contient aussi la fonction de projection, qui à partir de d'un point avec coordonées 3D, retourne le point projeté dont les coordonnées sont celles à l'écran.

\subsection{Implémentation du paquetage {\tt Scene}}
Le paquetage {\tt Scene} gère les éléments de la scène, à savoir la caméra (sa position, son orientation), et le modèle appelé maillage.
Après modification de l'orientation de la caméra, suite aux actions de l'utilisateur, les différentes matrices de rotation sont remises à jour.

\subsection{Implémentation du paquetage {\tt Ligne}}
Ce paquetage propose plusieurs algorithmes de tracé de ligne qui ne sont en réalité que des variantes du tracé fourni au départ.
Afin d'optimiser le nombre de pixels à dessiner, on cherche à ne dessiner que ceux étant situés à l'intérieur de l'écran.
Pour cela on utilise un algorithme de "clipping" : Liang-Barsky. Cet algorithme calculeth les intersections entre la ligne à dessiner et les bords de l'écran afin d'obtenir la portion visible de la ligne.

\subsection{Implémentation du paquetage {\tt Triangle}}
Ce nouveau paquetage propose une procédure de dessin de triangle plein. Pour cela, on utilise la méthode des coordonnées baricentriques.
Cela consiste à parcourir chaque pixel du carré englobant le triangle et à tester si le pixel est à l'intérieur du triangle ou non.
Encore une fois, on limite la boite englobante par les bords de l'écran afin de ne pas tracer des pixels inutiles.

\subsection{Implémentation du paquetage {\tt ZBuffer}}
Ce nouveau paquetage est un intermédiaire entre les algorithmes de tracé ({\tt Ligne} ou {\tt Triangle})
et le paquetage de dessin ({\tt Dessin}). Il permet de gérer la profondeur des pixels et d'obtenir un effet réaliste, une face arrière ne pouvant pas écraser une face plus en avant.
On utilise simplement un tableau qui associe à chaque pixel la profondeur courante.

\subsection{Implémentation du paquetage {\tt Params}}
Ce paquetage gère les paramètres globaux pour la configuration de l'affichage.

\subsection{Implémentation du paquetage {\tt Frame}}
Le calcul de la frame se déroule de la façon suivante : on efface l'écran, le tampon de profondeur. Pour chaque facette, on projette les points, et on vérifie qu'au moins un des points soit visible. Si le paramètre de luminosité est activé, on calcul la luminosité de la face, puis on affiche éventuellement la normale, enfin on dessine le triangle ou trois lignes suivant le mode de rendu.


% Fin implémentation ***************************************************



\section{Validation}

\subsection{Test du paquetage {\tt STL}}

Le programme de test du paquetage est \verb!test_stl.adb!. Il permet de charger un fichier et d'afficher le nombre de facettes lues ainsi que les coordonnées des sommets.

\subsection{Test du paquetage {\tt Algebre}}
Le programme de test du paquetage est \verb!test_algebre.adb!. Il permet d'effectuer quelques opérations sur les matrices et vecteurs et de s'assurer que les résultats sont cohérents.


\subsection{Test du programme}

Afin de tester le programme, nous avons plusieurs modèles à notre disposition tels q'un cube permettant de s'assurer que les primitives sont bien rendues ; une soucoupe volante permettant de tester la performance avec de nombreuses faces ; ainsi que Suzanne, une tête de singe, mascotte du logiciel de modélisation 3D Blender.


\section{Bibliographie}

Lire un fichier binaire en Ada \\
\url{http://www.infres.enst.fr/~pautet/Ada95/chap24.htm}

Dessiner un triangle avec la méthode barycentrique. \\
\url{courses.cms.caltech.edu/cs171/barycentric.pdf}

Découper les lignes. \\
\url{http://myweb.lmu.edu/dondi/share/cg/clipping.pdf} \\
\url{http://en.wikipedia.org/wiki/Line_clipping}

Rendu de lumière en OpenGL. \\
\url{http://www.glprogramming.com/red/chapter05.html#name7}





\end{document}
